


We will now focus on the spectra of operators and on the decomposition of the spectra of self-adjoint operators. The significance of spectra is that the axioms of quantum mechanics prescribe that the possible measurement values of an observable (which is, in particular, a self-adjoint operator) are those in the so-called spectrum of the operator.

A common task in almost any quantum mechanical problem that you might wish to solve is to determine the spectrum of some observable. This is usually the Hamiltonian, or energy operator, since the time evolution of a quantum system is governed by the exponential of the Hamiltonian, which is more practically determined by first determining its spectrum.

More often than not, it is not possible to determine the spectrum of an operator exactly (i.e.\ analytically). One then resorts to perturbation theory which consists in expressing the operator whose spectrum we want to determine as the sum of an operator whose spectrum can be determined analytically and another whose contribution is ``small'' in some sense to be made precise.



\subsection{Resolvent map and spectrum}

\bd
The \emph{resolvent map}\index{resolvent map} of an operator $A$ is the map
\bi{rrCl}
R_A\cl & \rho(A)& \to & \mathcal{L}(\mathcal{H})\\
& z & \mapsto & (A-z)^{-1},
\ei
where $\mathcal{L}(\mathcal{H})\equiv \mathcal{L}(\mathcal{H},\mathcal{H})$ and $\rho(A)$ is the \emph{resolvent set} of $A$, defined as
\bse
\rho(A):=\{z\in\C \mid (A-z)^{-1}\in \mathcal{L}(\mathcal{H})\}.
\ese
\ed

\br
Checking whether a complex number $z$ belongs to $\rho(A)$ may seem like a daunting task and, in general, it is. However, we will almost exclusively be interested in closed operators, and the closed graph theorem states that if $A$ is closed, then $(A-z)^{-1}\in \mathcal{L}(\mathcal{H})$ if, and only if, $A-z$ is bijective.
\er

\bd
The \emph{spectrum}\index{spectrum} of an operator $A$ is $\sigma(A):=\C\setminus\rho(A)$.
\ed

\bd
A complex number $\lambda\in\C$ is said to be an \emph{eigenvalue}\index{eigenvalue} of $A\cl\mathcal{D}_A\to\mathcal{H}$ if
\bse
\exists\, \psi \in \mathcal{D}_A\setminus \{0\} : \ A\psi = \lambda \psi.
\ese
Such an element $\psi$ is called an \emph{eigenvector}\index{eigenvector} of $A$ associated to the eigenvalue $\lambda$.
\ed

\bc
Let $\lambda\in\C$ be an eigenvalue of $A$. Then, $\lambda\in\sigma(A)$.
\ec

\bq
If $\lambda$ is an eigenvalue of $A$, then there exists $\psi\in\mathcal{D}_A\setminus\{0\}$ such that $A\psi = \lambda \psi$, i.e.\ \bse
(A-\lambda)\psi = 0.
\ese
Thus, $\psi\in\ker(A-\lambda)$ and hence, since $\psi\neq 0$, we have
\bse
\ker(A-\lambda)\neq\{0\}.
\ese
This means that $A-\lambda$ is not injective, hence not invertible and thus, $\lambda\notin\rho(A)$. Then, by definition, $\lambda\in\sigma(A)$.
\eq

\br
If $\mathcal{H}$ is finite-dimensional, then the converse of the above corollary holds ad hence, the spectrum coincides with the set of eigenvalues. However, in infinite-dimensional spaces, the spectrum of an operator contains more than just the eigenvalues of the operator.
\er

\subsection{The spectrum of a self-adjoint operator}

Recall that a self-adjoint operator is necessarily closed since $A=A^*$ implies $A=A^{**}$. While the following refinement of the notion of spectrum can be made in greater generality, we will primarily be interested in the case of self-adjoint operators.

\bd
Let $A$ be a self-adjoint operator. Then, we define
\ben[label=(\roman*)]
\item the \emph{pure point spectrum} of $A$
\bse
\sigma_{\mathrm{pp}}(A) := \{z\in\C\mid \ran(A-z)=\overline{\ran(A-z)}\neq\mathcal{H}\}
\ese
\item the \emph{point embedded in continuum spectrum} of $A$
\bse
\sigma_{\mathrm{pec}}(A) := \{z\in\C\mid \ran(A-z)\neq\overline{\ran(A-z)}\neq\mathcal{H}\}
\ese
\item the \emph{purely continuous spectrum} of $A$
\bse
\sigma_{\mathrm{pc}}(A) := \{z\in\C\mid \ran(A-z)\neq\overline{\ran(A-z)}=\mathcal{H}\}.
\ese
\een
\ed
These form a partition of $\sigma(A)$, i.e.\ they are pairwise disjoint and their union is $\sigma(A)$.
\bd
Let $A$ be a self-adjoint operator. Then, we further define
\ben[label=(\roman*)]
\item the \emph{point spectrum}\index{point spectrum} of $A$
\bse
\sigma_{\mathrm{p}}(A) := \sigma_{\mathrm{pp}}(A)\cup \sigma_{\mathrm{pec}}(A) = \{z\in\C\mid \overline{\ran(A-z)}\neq\mathcal{H}\}
\ese
\item the \emph{continuous spectrum}\index{continuous spectrum} of $A$
\bse
\sigma_{\mathrm{c}}(A) :=  \sigma_{\mathrm{pec}}(A)\cup  \sigma_{\mathrm{pc}}(A)=\{z\in\C\mid \ran(A-z)\neq\overline{\ran(A-z)}\}.
\ese
\een
\ed
Clearly, $\sigma_{\mathrm{p}}(A) \cup\sigma_{\mathrm{c}}(A) =\sigma(A)$ but, since $\sigma_{\mathrm{p}}(A) \cap\sigma_{\mathrm{c}}(A) = \sigma_{\mathrm{pec}}(A)$ is not necessarily empty, the point and continuous spectra do not form a partition of the spectrum in general.

\bl
Let $A$ be self-adjoint and let $\lambda$ be an eigenvalue of $A$. Then, $\lambda\in\R$.
\el

\bq
Let $\psi\in\mathcal{D}_A\setminus\{0\}$ be an eigenvector of $A$ associated to $\lambda$. By self-adjointness of $A$,
\bse
\lambda \langle \psi|\psi\rangle  =   \langle \psi|\lambda\psi\rangle = \langle \psi|A\psi\rangle = \langle A\psi|\psi\rangle =  \langle\lambda \psi|\psi\rangle= \overline{\lambda} \langle \psi|\psi\rangle.
\ese
Thus, we have
\bse
(\lambda-\overline{\lambda})\langle \psi|\psi\rangle =0 
\ese
and since $\psi\neq 0$, it follows that $\lambda=\overline{\lambda}$. That is, $\lambda\in\R$. 
\eq

\bt
If $A$ is a self-adjoint operator, then the elements of $\sigma_{\mathrm{p}}(A)$ are precisely the eigenvalues of $A$.
\et

\bq
\begin{itemize}
\item[($\Leftarrow$)] Suppose that $\lambda$ is an eigenvalue of $A$. Then, by self-adjointness of $A$,
\bse
\{0\}\neq \ker(A-\lambda)=\ker(A^*-\lambda) = \ker((A-\overline{\lambda})^*) = \ran(A-\overline{\lambda})^{\perp} = \ran(A-\lambda)^{\perp},
\ese
where we made use of our previous lemma. Hence, we have
\bse
\overline{ \ran(A-\lambda)} =  \ran(A-\lambda)^{\perp\perp} \neq \{0\}^{\perp} = \mathcal{H}
\ese
and thus, $\lambda\in\sigma_{\mathrm{p}}(A)$.

\item[($\Rightarrow$)] We now need to show that if $\lambda\in\sigma_{\mathrm{p}}(A)$, then $\lambda$ is an eigenvalue of $A$. By contraposition, suppose that $\lambda\in\C$ is not an eigenvalue of $A$. Note that if $\lambda$ is real, then $\lambda = \overline{\lambda}$ while if $\lambda$ is not real, then $\overline{\lambda}$ is not real. Hence, if $\lambda$ is not an eigenvalue of $A$, then neither is $\overline{\lambda}$. Therefore, there exists no non-zero $\psi$ in $\mathcal{D}_A$ such that $A\psi=\overline{\lambda}\psi$. Thus, we have
\bse
\{0\} = \ker(A-\overline{\lambda})= \ker(A^*-\overline{\lambda})= \ker((A-\lambda)^*) = \ran(A-\lambda)^{\perp}
\ese
and hence
\bse
\overline{\ran(A-\lambda)}=\ran(A-\lambda)^{\perp\perp} = \{0\}^{\perp}=\mathcal{H}.
\ese
Therefore, $\lambda\notin\sigma_{\mathrm{p}}(A)$.\qedhere
\end{itemize}
\eq

\br
The \emph{contrapositive}\index{contrapositive} of the statement $P\Rightarrow Q$ is the statement $\neg Q\Rightarrow \neg P$, where the symbol $\neg$ denotes logical negation. A statement and its contrapositive are logically equivalent and ``proof by contraposition'' simply means ``proof of the contrapositive''.
\er

\subsection{Perturbation theory for point spectra of self-adjoint operators}

Before we move on to perturbation theory, we will need some preliminary definitions. First, note that if $\psi$ and $\varphi$ are both eigenvectors of an operator $A$ associated to some eigenvalue $\lambda$, then, for any $z\in \C$, the vector $z\psi+\varphi$ is either zero or it is again an eigenvector of $A$ associated to $\lambda$. 

\bd
Let $A$ be an operator and let $\lambda$ be an eigenvalue of $A$.
\ben[label=(\roman*)]
\item The \emph{eigenspace}\index{eigenspace} of $A$ associated to $\lambda$ is
\bse
\Eig_A(\lambda) := \{\psi \in \mathcal{D}_A\mid A\psi = \lambda \psi\}.
\ese
\item The eigenvalue $\lambda$ is said to be \emph{non-degenerate} if $\dim \Eig_A(\lambda)=1$, and \emph{degenerate} if $\dim \Eig_A(\lambda)>1$.  
\item The \emph{degeneracy}\index{degeneracy} of $\lambda$ is $\dim \Eig_A(\lambda)$.
\een
\ed
\br
Of course, it is possible that $\dim \Eig_A(\lambda)=\infty$ in general. However, in this section, we will only consider operators whose eigenspaces are finite-dimensional. 
\er

\bl
\label{lem:EigenvectorsOrthogonalDistinctEigenvalues}
Eigenvectors associated to distinct eigenvalues of a self-adjoint operator are orthogonal. 
\el

\bq
Let $\lambda,\lambda'$ be distinct eigenvalues of a self-adjoint operator $A$ and let $\psi,\varphi\in\mathcal{D}_A\setminus\{0\}$ be eigenvectors associated to $\lambda$ and $\lambda'$, respectively. As $A$ is self-adjoint, we already know that $\lambda,\lambda'\in\R$. Then, note that
\bi{rCl}
(\lambda-\lambda')\langle \psi|\varphi\rangle & = & \lambda\langle \psi|\varphi\rangle-\lambda'\langle \psi|\varphi\rangle\\
& = & \langle \lambda\psi|\varphi\rangle-\langle \psi|\lambda'\varphi\rangle\\
& = & \langle A\psi|\varphi\rangle-\langle \psi|A\varphi\rangle \\
& = & \langle \psi|A\varphi\rangle-\langle \psi|A\varphi\rangle \\
& = & 0.
\ei
Since $\lambda-\lambda'\neq 0$, we must have $\langle \psi|\varphi\rangle =0$.
\eq

\subsubsection*{A. \ Unperturbed spectrum}

Let $H_0$ be a self-adjoint operator whose eigenvalues and eigenvectors are known and satisfy
\bse
H_0e_{n\delta}=h_ne_{n\delta},
\ese
where
\begin{itemize}
\item the index $n$ varies either over $\N$ or some finite range $1,2,\ldots,N$
\item the real numbers $h_n$ are the eigenvalues of $H_0$
\item the index $\delta$ varies over the range $1,2,\ldots,d(n)$, with $d(n):=\dim \Eig_{H_0}(h_n)$
\item for each fixed $n$, the set 
\bse
\{e_{n\delta}\in\mathcal{D}_{H_0}\mid 1\leq \delta\leq d(n)\}
\ese
is a linearly independent subset (in fact, a Hamel basis) of $\Eig_{H_0}(h_n)$.
\end{itemize}
Note that, since we are assuming that all eigenspaces of $H_0$ are finite-dimensional, $\Eig_{H_0}(h_n)$ is a sub-Hilbert space of $\mathcal{H}$ and hence, for each fixed $n$, we can choose the $e_{n\delta}$ so that
\bse
\langle e_{n\alpha}|e_{n\beta}\rangle = \delta_{\alpha\beta}.
\ese
In fact, thanks to our previous lemma, we can choose the eigenvectors of $H_0$ so that
\bse
\langle e_{n\alpha}|e_{m\beta}\rangle = \delta_{nm}\delta_{\alpha\beta}.
\ese

Let $W\cl\mathcal{D}_{H_0}\to\mathcal{H}$ be a not necessarily self-adjoint operator. Let $\lambda\in(-\varepsilon,\varepsilon)\subseteq \R$ and consider the real one-parameter family of operators $\{H_{\lambda}\mid\lambda\in(-\varepsilon,\varepsilon)\}$, where
\bse
H_{\lambda} := H_0+\lambda W.
\ese
Further assume that $H_{\lambda}$ is self-adjoint for all $\lambda\in(-\varepsilon,\varepsilon)$. Recall, however, that this assumption does \emph{not} force $W$ to be self-adjoint.

We seek to understand the eigenvalue equation for $H_{\lambda}$,
\bse
H_{\lambda} e_{n\delta}(\lambda)=h_{n\delta}(\lambda)e_{n\delta}(\lambda),
\ese
by exploiting the fact that it coincides with the eigenvalue equation for $H_0$ when $\lambda=0$. In particular, we will be interested in the lifting of the degeneracy of $h_n$ (for some fixed $n$) once the perturbation $W$ is ``switched on'', i.e.\ when $\lambda\neq 0$. Indeed, it is possible, for instance, that while the two eigenvectors $e_{n1}$ and $e_{n2}$ are associated to the same (degenerate) eigenvalue $h_n$ of $H_0$, the ``perturbed'' eigenvectors $e_{n1}(\lambda)$ and $e_{n2}(\lambda)$ may be associated to \emph{different} eigenvalues of $H_{\lambda}$. Hence the reason why we added a $\delta$-index to the eigenvalue in the above equation. Of course, when $\lambda =0$, we have $h_{n\delta}(\lambda)=h_n$ for all $\delta$.


\subsubsection*{B. \ Formal power series ansatz\footnote{German for ``educated guess''.}}

In order to determine $h_{n\delta}(\lambda)$ and $e_{n\delta}(\lambda)$, we make, for both, the following ansatz
\bi{rCl}
h_{n\delta}(\lambda) & =: & h_n + \lambda\theta_{n\delta}^{(1)} + \lambda^2\theta_{n\delta}^{(2)}+\mathcal{O}(\lambda^3)\\
e_{n\delta}(\lambda) & =: & e_{n\delta} + \lambda\epsilon_{n\delta}^{(1)} + \lambda^2\epsilon_{n\delta}^{(2)}+\mathcal{O}(\lambda^3),
\ei
where $\theta_{n\delta}^{(1)},\theta_{n\delta}^{(2)}\in \R$ and $\epsilon_{n\delta}^{(1)},\epsilon_{n\delta}^{(2)}\in\mathcal{D}_{H_0}$.

\br
Recall that the \emph{Big $\mathcal{O}$ notation} is defined as follows. If $f$ and $g$ are functions $I\subseteq\R\to\R$ and $a\in I$, then we write
\bse
f(x)=\mathcal{O}(g(x)) \quad \text{as }x\to a
\ese
to mean
\bse
\exists \, k,M>0 : \forall \, x\in I : \ 0<|x-a|<k \ \Rightarrow \ |f(x)|<M|g(x)|.
\ese
The qualifier ``as $x\to a$'' can be omitted when the value of $a$ is clear from the context. In our expressions above, we obviously have ``as $\lambda \to 0$''.
\er

\subsubsection*{C. \ Fixing phase and normalisation of perturbed eigenvectors}

Eigenvectors in a complex vector space are only defined up to a complex scalar or, alternatively, up to phase and magnitude. Hence, we impose the following conditions relating the perturbed eigenvalues and eigenvectors to the unperturbed ones.

We require, for all $\lambda\in(-\varepsilon,\varepsilon)$, all $n$ and all $\delta$,
\ben[label=(\roman*)]
\item $\Im\langle e_{n\delta}|e_{n\delta}(\lambda)\rangle = 0$
\item $\|e_{n\delta}(\lambda)\|^2=1$.
\een
Inserting the formal power series ansatz into these conditions yields
\ben[label=(\roman*)]
\item $\Im\langle e_{n\delta}|\epsilon^{(k)}_{n\delta}\rangle = 0$ for $k=1,2,\ldots$
\item $0 = 2\lambda\Re\langle e_{n\delta}|\epsilon^{(1)}_{n\delta}\rangle +\lambda^2\bigl(2\Re\langle e_{n\delta}|\epsilon^{(2)}_{n\delta}\rangle +\|\epsilon^{(1)}_{n\delta}\|^2\bigr)+ \mathcal{O}(\lambda^3)$.
\een
Since (ii) holds for all $\lambda\in(-\varepsilon,\varepsilon)$, we must have 
\bse
\Re\langle e_{n\delta}|\epsilon^{(1)}_{n\delta}\rangle=0, \qquad 2\Re\langle e_{n\delta}|\epsilon^{(2)}_{n\delta}\rangle +\|\epsilon^{(1)}_{n\delta}\|^2=0. 
\ese
Since we know from (i) that $\Im\langle e_{n\delta}|\epsilon^{(1)}_{n\delta}\rangle = 0$ and $\Im\langle e_{n\delta}|\epsilon^{(2)}_{n\delta}\rangle = 0$, we can conclude
\bse
\langle e_{n\delta}|\epsilon^{(1)}_{n\delta}\rangle=0, \qquad  \langle e_{n\delta}|\epsilon^{(2)}_{n\delta}\rangle =-\tfrac{1}{2}\|\epsilon^{(1)}_{n\delta}\|^2. 
\ese
That is, $\epsilon^{(1)}_{n\delta}$ is orthogonal to $e_{n\delta}$ and
\bse
\epsilon^{(2)}_{n\delta} = -\tfrac{1}{2}\|\epsilon^{(1)}_{n\delta}\|^2e_{n\delta} + e
\ese
for some $e\in \lspan(\{e_{n\delta}\})^{\perp}$.

\subsubsection*{D. \ Order-by-order decomposition of the perturbed eigenvalue problem}

Let us insert our formal power series ansatz into the perturbed eigenvalue equation. On the left-hand side, we find
\bi{rCl}
H_{\lambda}e_{n\delta}(\lambda) & = & (H_0+\lambda W)(e_{n\delta} + \lambda\epsilon_{n\delta}^{(1)} + \lambda^2\epsilon_{n\delta}^{(2)}+\mathcal{O}(\lambda^3))\\
& = & H_0e_{n\delta} + \lambda (We_{n\delta}+H_0\epsilon_{n\delta}^{(1)} ) + \lambda^2 (W\epsilon_{n\delta}^{(1)} +H_0\epsilon_{n\delta}^{(2)} )+\mathcal{O}(\lambda^3),
\ei
while, on the right-hand side, we have
\bi{rCl}
h_{n\delta}(\lambda)e_{n\delta}(\lambda) & = & (h_n + \lambda\theta_{n\delta}^{(1)} + \lambda^2\theta_{n\delta}^{(2)}+\mathcal{O}(\lambda^3)) (e_{n\delta} + \lambda\epsilon_{n\delta}^{(1)} + \lambda^2\epsilon_{n\delta}^{(2)}+\mathcal{O}(\lambda^3))\\
& = & h_ne_{n\delta}+ \lambda (h_n\epsilon_{n\delta}^{(1)}+\theta_{n\delta}^{(1)}e_{n\delta})+\lambda^2(h_n\epsilon_{n\delta}^{(2)}+\theta_{n\delta}^{(1)}\epsilon_{n\delta}^{(1)}+\theta_{n\delta}^{(2)}e_{n\delta})+ \mathcal{O}(\lambda^3).
\ei
Comparing terms order-by-order yields
\bi{rCl}
(H_0-h_n)e_{n\delta} & = & 0 \\
(H_0-h_n)\epsilon_{n\delta}^{(1)} & = & -(W -\theta_{n\delta}^{(1)}) e_{n\delta}\\
(H_0-h_n)\epsilon_{n\delta}^{(2)} & = & -(W -\theta_{n\delta}^{(1)}) \epsilon_{n\delta}^{(1)}+\theta_{n\delta}^{(2)}e_{n\delta}.
\ei
Of course, one may continue this expansion up to the desired order. Note that the zeroth order equation is just our unperturbed eigenvalue equation. 

\subsubsection*{E. \ First-order correction}
To extract information from the first-order equation, let us project both sides onto the unperturbed eigenvectors $e_{n\alpha}$ (i.e.\ apply $\langle e_{n\alpha}|\,\cdot\,\rangle$ to both sides). This yields
\bse
\langle e_{n\alpha}|(H_0-h_n)\epsilon_{n\delta}^{(1)} \rangle = -\langle e_{n\alpha}|(W -\theta_{n\delta}^{(1)}) e_{n\delta}\rangle.
\ese
By self-adjointness of $H_0$, we have
\bse
\langle e_{n\alpha}|(H_0-h_n)\epsilon_{n\delta}^{(1)} \rangle =\langle (H_0-h_n)^*e_{n\alpha}|\epsilon_{n\delta}^{(1)} \rangle =  \langle (H_0-h_n)e_{n\alpha}|\epsilon_{n\delta}^{(1)} \rangle = 0.
\ese
Therefore,
\bse
0 = -\langle e_{n\alpha}|W e_{n\delta}\rangle + \langle e_{n\alpha}|\theta_{n\delta}^{(1)} e_{n\delta}\rangle = -\langle e_{n\alpha}|W e_{n\delta}\rangle + \theta_{n\delta}^{(1)} \delta_{\alpha\delta} 
\ese
and thus, the first-order eigenvalue correction is 
\bse
\theta_{n\delta}^{(1)} = \langle e_{n\delta}|W e_{n\delta}\rangle .
\ese

Note that the right-hand side of the first-order equation is now completely known and hence, if $H_0-h_n$ were invertible, we could determine $\epsilon_{n\delta}^{(1)}$ immediately. However, this is only possible if the unperturbed eigenvalue $h_n$ is non-degenerate. More generally, we proceed as follows. Let $E:=\Eig_{H_0}(h_n)$. Then, we can rewrite the right-hand side of the first-order equation as
\bi{rCl}
-(W -\theta_{n\delta}^{(1)}) e_{n\delta} & = & -\id_{\mathcal{H}}(W -\theta_{n\delta}^{(1)}) e_{n\delta}\\
& = & -(\mathrm{P}_{\! E}+\mathrm{P}_{\! E^{\perp}})(W -\theta_{n\delta}^{(1)}) e_{n\delta}\\
& = & -\sum_{\beta=1}^{d(n)}\langle e_{n\beta}|(W -\theta_{n\delta}^{(1)}) e_{n\delta}\rangle e_{n\beta} -\mathrm{P}_{\! E^{\perp}}We_{n\delta} +\theta_{n\delta}^{(1)}\mathrm{P}_{\! E^{\perp}} e_{n\delta}\\
& = &  -\mathrm{P}_{\! E^{\perp}}W e_{n\delta}
\ei
so that we have $(H_0-h_n)\epsilon_{n\delta}^{(1)}\in E^{\perp}$. Note that the operator
\bse
\mathrm{P}_{\! E^{\perp}}\circ(H_0-h_n)\cl E^{\perp} \to E^{\perp} 
\ese
is invertible. Hence, the equation
\bse
\mathrm{P}_{\! E^{\perp}}(H_0-h_n)\mathrm{P}_{\! E^{\perp}}\epsilon_{n\delta}^{(1)}  =  -\mathrm{P}_{\! E^{\perp}}W e_{n\delta} 
\ese
is solved by
\bse
\mathrm{P}_{\! E^{\perp}}\epsilon_{n\delta}^{(1)} = -\mathrm{P}_{\! E^{\perp}}(H_0-h_n)^{-1}\mathrm{P}_{\! E^{\perp}}W e_{n\delta} .
\ese
The ``full'' eigenvector correction $\epsilon_{n\delta}^{(1)}$ is given by
\bse
\id_{\mathcal{H}}\epsilon_{n\delta}^{(1)} = (\mathrm{P}_{\! E}+\mathrm{P}_{\! E^{\perp}})\epsilon_{n\delta}^{(1)} = \sum_{\beta=1}^{d(n)}c_{\delta\beta}e_{n\beta} -\mathrm{P}_{\! E^{\perp}}(H_0-h_n)^{-1}\mathrm{P}_{\! E^{\perp}}W e_{n\delta} ,
\ese
where the coefficients $c_{\delta\beta}$ cannot be fully determined at this order in the perturbation. What we do know is that our previous fixing of the phase and normalisation of the perturbed eigenvectors implies that $\epsilon_{n\delta}^{(1)}$ is orthogonal to $e_{n\delta}$, and hence we must have $c_{\delta\delta}=0$.

\subsubsection*{F. \ Second-order eigenvalue correction}

Here we will content ourselves with calculating the second-order correction to the eigenvalues only, since that is the physically interesting formula. As before, we proceed by projecting both sides of the second-order equation onto an unperturbed eigenvector, this time specifically $e_{n\delta}$. We find

\bse
\langle e_{n\delta}|(H_0-h_n)\epsilon_{n\delta}^{(2)} \rangle = -\langle e_{n\delta}| W\epsilon_{n\delta}^{(1)}\rangle -\theta_{n\delta}^{(1)} \langle e_{n\delta}| \epsilon_{n\delta}^{(1)}\rangle+\theta_{n\delta}^{(2)}\langle e_{n\delta}| e_{n\delta}\rangle.
\ese
Noting, as before, that
\bse
\langle e_{n\delta}|(H_0-h_n)\epsilon_{n\delta}^{(2)} \rangle = 0
\ese
and recalling that $\langle e_{n\delta}| \epsilon_{n\delta}^{(1)}\rangle=0$ and $\langle e_{n\delta}| e_{n\delta}\rangle=1$, we have
\bse
\theta_{n\delta}^{(2)} = \langle e_{n\delta}| W\epsilon_{n\delta}^{(1)}\rangle .
\ese
Plugging in our previous expression for $\epsilon_{n\delta}^{(1)}$ yields
\bi{rCl}
\theta_{n\delta}^{(2)} & = & \biggl\langle e_{n\delta}\ \bigg| \ W \sum_{\beta=1}^{d(n)}c_{\delta\beta}e_{n\beta} -W\mathrm{P}_{\! E^{\perp}}(H_0-h_n)^{-1}\mathrm{P}_{\! E^{\perp}}W e_{n\delta}\biggr\rangle \\
& = & \sum_{\beta=1}^{d(n)} c_{\delta\beta}\langle e_{n\delta}|We_{n\beta}\rangle  -\langle e_{n\delta}|W\mathrm{P}_{\! E^{\perp}}(H_0-h_n)^{-1}\mathrm{P}_{\! E^{\perp}}W e_{n\delta}\rangle \\
& = & \sum_{\beta=1}^{d(n)} c_{\delta\beta}\theta_{n\beta}^{(1)}\delta_{\delta\beta}  -\langle e_{n\delta}|W\mathrm{P}_{\! E^{\perp}}(H_0-h_n)^{-1}\mathrm{P}_{\! E^{\perp}}W e_{n\delta}\rangle\\
& = & -\langle e_{n\delta}|W\mathrm{P}_{\! E^{\perp}}(H_0-h_n)^{-1}\mathrm{P}_{\! E^{\perp}}W e_{n\delta}\rangle
\ei
since $c_{\delta\delta}=0$. One can show that the eigenvectors of $H_0$ (or any other self-adjoint operator) form an orthonormal basis of $\mathcal{H}$. In particular, this implies than we can decompose the identity operator on $\mathcal{H}$ as
\bse
\id_{\mathcal{H}} = \sum_{n=1}^{\infty}\sum_{\beta=1}^{d(n)}\langle e_{n\beta}|\,\cdot\,\rangle e_{n\beta}.
\ese
By inserting this appropriately into our previous expression for $\theta_{n\delta}^{(2)}$, we obtain
\bse
\theta_{n\delta}^{(2)} = - \sum_{\substack{m=1\\m\neq n}}^{\infty}\sum_{\beta=1}^{d(m)}\frac{|\langle e_{m\beta}|We_{n\delta}\rangle|}{h_m-h_n}.
\ese
Putting everything together, we have the following second-order expansion of the perturbed eigenvalues
\bi{rCl}
h_{n\delta}(\lambda) & = & h_n + \lambda \theta_{n\delta}^{(1)}+\lambda^2\theta_{n\delta}^{(2)} +\mathcal{O}(\lambda^3)\\
& = & h_n +\lambda \langle e_{n\delta}|We_{n\delta}\rangle - \lambda^2\sum_{\substack{m=1\\m\neq n}}^{\infty}\sum_{\beta=1}^{d(m)}\frac{|\langle e_{m\beta}|We_{n\delta}\rangle|}{h_m-h_n}+\mathcal{O}(\lambda^3).
\ei

\br
Note that, while the first-order correction to the perturbed $n\delta$ eigenvalue only depends on the unperturbed $n\delta$ eigenvalue and eigenvector, the second-order correction draws information from \emph{all} the unperturbed eigenvalues and eigenvectors. Hence, if we try to approximate a relativistic system as a perturbation of a non-relativistic system, then the second-order corrections may be unreliable.
\er

























